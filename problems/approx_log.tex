
\documentclass{article}
\begin{document}

\section*{Using a function to approximate the logarithm}
An approximation to the natural logarithm is given by
\begin{equation}
\log(1 + x) = 
\lim_{n \rightarrow \infty}
\sum_{i=1}^n \frac{1}{i}\left(\frac{x}{1+x}\right)^{i}
\end{equation}
\newline
Otherwise, simply:
\begin{equation}
\log(1 + x) = 
\sum_{i=1}^{\infty} \frac{1}{i}\left(\frac{x}{1+x}\right)^{i}
\end{equation}
\newline
Write a python function that verifies that
\begin{equation}
\log(1.1) = 0.09531
\end{equation}
\newline
You will want to replace `infinity' with just a large number, like 1000. You can
even set your program to stop when the difference between two successive
computations is small enough.
\newline
\newline
\subsection*{Using the \texttt{math} module}
There's actually a much simpler (and more efficient) way to calculate the
logarithm:

\begin{verbatim}
import math
print math.log(1.1)
\end{verbatim}

\end{document}
