
\documentclass{article}
\usepackage{graphicx}
\begin{document}
\section*{Simulation of Heat Conduction}
Heat conduction in two dimensions is described by the following differential equation:
\begin{equation}
k\left(\frac{\partial^{2} T}{\partial x^{2}} + 
\frac{\partial^{2} T}{\partial y^{2}}\right)
 = \rho C_{p} \frac{dT}{dt}
\newline
\end{equation}
The objectives are:
\begin{enumerate}
\item{To discretize and solve the equation over a square grid with some 
		boundary conditions}
\item{To produce a contour plot of the steady-state temperature distribution}
\item{To produce an animation to visualize the conduction of heat
		over time}
\item{To write a Makefile that automates the above tasks and updates the document
		\texttt{paper.pdf}}
\end{enumerate}
The simulation has already been implemented in the script \texttt{heat.py}, 
though you are welcome to do the same yourself. Run the script and it 
will save the results to the directory. The docstring explains how the results 
are stored. \newline \\
Write another script \texttt{post\_process.py} which will generate
and store the temperature contours at every time step. You can make a movie out
of the images using the \texttt{movie} function. Use the \texttt{help}
command to find out how it works:

\begin{verbatim}
>>> from scitools.easyviz import movie
>>> help(movie)
\end{verbatim}


\end{document}

 

