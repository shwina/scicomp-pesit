\documentclass{article}
\usepackage{graphicx}
\begin{document}

\section*{A Brief Introduction to making Iterative Fractals}
Ever looked closely at a broccoli?
\begin{figure}[h]
\begin{center}
\includegraphics[height=100pt]{../pictures/img1.jpg}
\end{center}
\end{figure}
\newline \\ No, closer
\begin{figure}[h]
\begin{center}
\includegraphics[height=100pt]{../pictures/img2.jpg}
\end{center}
\end{figure}
\newline \\ Even closer
\begin{figure}[h]
\begin{center}
\includegraphics[height=120pt]{../pictures/img3.jpg}
\end{center}
\end{figure}
\newline \\ What you just saw is a fractal. It is basically an image or a \emph{mapping} of a particular mathematical function. These are found in nature a lot. Snowflakes are fractals.
\begin{figure}[h]
\begin{center}
\includegraphics[height=180pt]{../pictures/img4.jpg}
\end{center}
\end{figure}
\newline \\ There are several mathematical techniques to obtain fractals, most of them involve complex numbers and complex functions.
\newline \\ Without going too deep into how the math behind the fractal actually works, I will try to explain how to obtain a fractal of the Mandelbrot Set. It looks something like this:
\begin{figure}[h]
\begin{center}
\includegraphics[height=200pt]{../pictures/99.png}
\end{center}
\end{figure}
\newline
\newline \\ Iterative fractals work in a way where you use some complex function, and keep 
feeding back the results into the same function to get the next set of points.
The complex function used in the Mandelbrot Set is
\begin{center} $Z_{n+1} = Z_n + c$ \end{center}
All terms are complex numbers. And the plot is on the complex plane. If 
you don't know about complex numbers, dont fret. Just pretend that these complex numbers 
just contain information about \texttt{(x,y)} co-ordinates. 
\newline \\ $Z_n$ contains the results of the previous computation (initially zero), $c$ is a complex matrix 
containing all the points on the plot we need. $Z_{n+1}$ is the next set of plots.
\newline \\ If we iterate over and over again, we can get a more and more detailed Mandelbrot Fractal.
To make a movie out of the images stored over every iteration, open a python shell in the folder and do
\newline \\ \texttt{from scitools.easyviz import movie}
\newline \\ \texttt{movie('*.png', encoder='mencoder', fps = 10)}
\subsection*{Additional Exercise}
If you have way too much time, you could write an obfuscated code for your fractal. Obfuscated code is what people write when they don't 
feel loved. This one is for the Mandelbrot Set. Check it out.
\begin{figure}[h]
\begin{center}
\includegraphics[height=180pt]{../pictures/img5.png}
\end{center}
\end{figure}
\newline
\begin{figure}[h]
\begin{center}
\includegraphics[height=200pt]{../pictures/M.jpg}
\end{center}
\end{figure}
\newline \\ You can read more about the guy who created the above code here:
\newline \\ \texttt{http://preshing.com/20110926/high-resolution-mandelbrot-in-obfuscated-python}
\end{document}
